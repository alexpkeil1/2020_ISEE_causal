\begin{frame} [c]{Policy definition of a cause\footnote{This is packed and will be formalized}}
	Exposure \textbf{causes} an outcome if a manipulation\footnote{e.g. a policy could include random assignment to treatment/control arms} of the exposure (e.g. via intervention) would change the outcome\footnote{Robins and Greenland (2000) \emph{J. Am. Stat. Assoc.}}
\end{frame}

%\begin{frame} [c]{Aside: can a state be a cause?}
%	\only<1>{ Debate with resolute camps: can obesity be a cause?\footnote{Vandenbroucke et al (2016) \emph{Int. J Epid.} + 8 commentaries }}
%	\only<2>{For today, no\footnote{Under the policy definition, the ``state'' determines the policy (e.g. if obese <state>, then restrict caloric intake <policy>). I mostly just don't want to talk about this the policy definition can absorb most of what we mean when we consider states as causes, provided that we collect good data - see also Vanderweele and Robinson (2014) \emph{Epidemiology}.}}
%	\only<2>{For today, no}
%\end{frame}


\begin{frame} [c]{Policy and decisions}
	\begin{itemize}
		\visible<1->{\item Causal effects contrast one policy with another\footnote{Or define a function in the case of continuous policies like dose-responses}}
		      \visible<2->{\item Causal effect estimation allows ``optimal'' policy choices}
		      \visible<3->{\item Epidemiologic causal inference is about improving \textbf{decisions} for groups in context}
	\end{itemize}
\end{frame}


\begin{frame}{Motivating example: coal fired power plants and cognitive development}
	\begin{itemize}
		\item Burning coal for energy produces many byproducts, including a mixture of air toxics
		\item Some of these have known detrimental effects on early-life cognitive outcomes
		\item Closure of coal-fired plants has been associated with improvements in cognitive outcomes
		\item (So that I can share the data) I performed a simulation study of exposure to coal-fired power plant emissions and mental development index
	\end{itemize}
\end{frame}

\begin{frame}{Motivating example: coal fired power plants and cognitive development}
  \begin{columns}
    \begin{column}[t]{.4\textwidth}
      \begin{itemize}
      \item Simulated N=2020 3 yr olds
      \item Annual mean residential air exposures (ug/m$^3$)
      \item $W1,W2$ associated with $Y$ via unmeasured, historical factors (education, racism)
      \item Coal plants near urban centers, and race associated with proximity to coal plants via redlining
    \end{itemize}
    \end{column}
    \begin{column}[t]{.7\textwidth}
  {\footnotesize
	  \begin{tikzpicture}[>=latex, line join=bevel, very thick]
	     %
             \node[](coal){\begin{tabular}{c}Proximity to \\ coal fired plant\end{tabular}};
             \node[above right=of coal, shift={(0,-1)}](a){Arsenic $(A)$};
             \node[right=of coal](b){Beryllium $(B)$};
             \node[below right=of coal, shift={(0,1)}](c){Cadmium $(C)$};
             \node[ right=of b] (y) {\begin{tabular}{c}Mental development\\ index $(Y)$\end{tabular}};
             \node[above =of a] (w1) {\begin{tabular}{c}\\Black race $(W2)$\end{tabular}};
             \node[above =of w1, shift={(0,-1)}] (w2) {\begin{tabular}{c}Urbanicity $(W1)$\end{tabular}};
             %
             \path[-latex] (coal) edge[] (a);  
             \path[-latex] (coal) edge[] (b);  
             \path[-latex] (coal) edge[] (c);  
             \path[-latex] (a) edge[] (y);  
             \path[-latex] (b) edge[] (y);  
             \path[-latex] (c) edge[] (y);  
             \path[-latex] (a) edge[] (y);  
             \path[latex-latex] (w1) edge[style=dashed] (y);  
             \path[latex-latex] (w1) edge[style=dashed] (coal);  
             \path[-latex] (w2) edge[bend right] (coal);  
             \path[latex-latex] (w2) edge[style=dashed, bend left] (y);  
          \end{tikzpicture} 
}
    \end{column}
  \end{columns}
\end{frame}


\begin{frame}{Motivating example: coal fired power plants and cognitive development}
(Possible) Causal questions of interest:
    \begin{description}
      \item[Causal independent exposure-response] How does the population average MDI change as we increase Arsenic but hold other exposures constant
      \item[Causal joint exposure-response] How does the population average MDI change as we increase all exposures by the same amount
      \item[Causal attributable difference] How does the population average MDI change after we eliminate all/some exposures?
      \item[Causal generalized impact difference] How does the population average MDI change after we reduce all/some exposures by some policy-relevant amount?
    \end{description}
\end{frame}


\begin{frame}{A note on study questions}
Causal inference allows us to ask questions with answers that make sense to people who don't know what regression is, without sacrificing rigor
\bigskip

I always encourage students to ask causal questions because it helps focus and choose methods even when causality is hopeless.

\end{frame}


